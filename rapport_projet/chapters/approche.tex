\section{Approche prise pour la conception}

\subsection{Introduction}
Ce chapitre détaille l'architecture logicielle retenue pour répondre aux contraintes de décentralisation, ainsi que les modèles comportementaux des agents (drones).

\subsection{Architecture du système}
L'architecture repose sur un modèle \textit{Edge Computing} pur et complètement décentralisé, où chaque drone est un nœud de calcul indépendant. Il n'y a pas de base de données partagée ni de service coordinateur central.

Le système implémente une coordination peer-to-peer avec :
\begin{itemize}
    \item \textbf{Communication MQTT} : Chaque drone publie sa position et écoute les autres
    \item \textbf{Leadership automatique} : Élection du leader basée sur l'ID minimum
    \item \textbf{Tolérance aux pannes} : Auto-réorganisation en cas de défaillance
    \item \textbf{Zone émergente} : L'enveloppe convexe est calculée par le leader uniquement
\end{itemize}

La zone de vol n'est pas définie par des coordonnées GPS fixes, mais est \textbf{émergente} et calculée de manière distribuée : seul le drone leader calcule l'enveloppe convexe des positions de tous les drones actifs à un instant $t$.

\begin{figure}[H]
    \centering
    \includegraphics[width=0.8\textwidth]{img/architecture.png}
    \caption{Architecture logique d'un drone}
    \label{fig:arch}
\end{figure}

Comme illustré dans la Figure \ref{fig:arch}, chaque drone possède ses propres modules de communication, d'estimation de zone et de logique de géofencing.

\begin{figure}[H]
    \centering
    \includegraphics[width=0.9\textwidth]{img/class_diagram.png}
    \caption{Diagramme de classes du système}
    \label{fig:classes}
\end{figure}

Le diagramme de classes (Figure \ref{fig:classes}) met en évidence la séparation entre la gestion des données brutes (\texttt{DroneInfo}, \texttt{Position}) et la logique métier (\texttt{LocalZoneEstimator}, \texttt{GeofencingLogic}).

\subsection{Diagrammes de cas d'utilisations}
Les principaux cas d'utilisation pour un drone autonome dans un système décentralisé sont :
\begin{itemize}
    \item \textbf{Diffuser sa position :} Publier périodiquement ses coordonnées via MQTT (toutes les 3 secondes).
    \item \textbf{Écouter les autres drones :} Recevoir et stocker les positions des autres drones actifs.
    \item \textbf{Élire un leader :} Participer au processus d'élection automatique basé sur l'ID minimum.
    \item \textbf{Calculer la zone globale :} Si leader, calculer l'enveloppe convexe de tous les drones.
    \item \textbf{Publier la zone :} Si leader, diffuser la zone de surveillance calculée.
    \item \textbf{Auto-nettoyage :} Supprimer les drones inactifs (timeout de 15 secondes).
    \item \textbf{Gérer les défaillances :} Réélire un nouveau leader si l'actuel disparaît.
\end{itemize}

\subsection{Diagrammes de séquences}
Le processus de maintien de la cohésion suit une boucle de rétroaction continue.

\begin{figure}[H]
    \centering
    \includegraphics[width=0.9\textwidth]{img/sequence.png}
    \caption{Séquence de diffusion et de correction de position}
    \label{fig:seq}
\end{figure}

La Figure \ref{fig:seq} montre comment la réception d'une position déclenche la mise à jour de l'estimateur de zone (\texttt{Zone Estimator}), qui informe ensuite la logique de géofencing pour appliquer une éventuelle correction de trajectoire.

\subsection{Diagrammes d'états}
Chaque drone évolue selon un automate à états finis pour gérer son statut au sein de l'essaim.

\begin{figure}[H]
    \centering
    \includegraphics[width=0.7\textwidth]{img/states.png}
    \caption{Diagramme d'états du drone}
    \label{fig:state}
\end{figure}

L'état critique est \texttt{Operating}, qui se subdivise en sous-états de surveillance (\texttt{InZone}, \texttt{AtBorder}, \texttt{OutOfZone}).

\subsection{Avantages de l'approche décentralisée}
L'architecture complètement décentralisée apporte plusieurs bénéfices critiques :
\begin{itemize}
    \item \textbf{Résilience maximale :} Aucun point de défaillance unique, pas de service centralisé critique
    \item \textbf{Auto-organisation :} Les drones s'organisent automatiquement sans intervention externe
    \item \textbf{Scalabilité dynamique :} Ajout/suppression de drones à chaud sans reconfiguration
    \item \textbf{Leadership adaptatif :} Élection automatique et ré-élection en cas de panne du leader
    \item \textbf{Performance optimisée :} Seul le leader effectue les calculs coûteux (enveloppe convexe)
    \item \textbf{Tolérance aux pannes :} La mission continue même avec des défaillances multiples
\end{itemize}

La zone se "referme" automatiquement autour des membres restants lors de la perte d'un drone, sans nécessiter de recalibrage manuel.
