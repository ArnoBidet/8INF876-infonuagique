\section{Conclusion}

Au terme de ce projet, nous avons pu implémenter et valider un système de géofencing complètement décentralisé pour essaim de drones, démontrant la faisabilité d'une coordination peer-to-peer sans infrastructure critique.

\textbf{Ce que nous avons accompli :}
\begin{itemize}
    \item \textbf{Architecture zéro-défaillance :} Élimination de tout service centralisé critique
    \item \textbf{Leadership adaptatif :} Mécanisme d'élection automatique et transparente des leaders
    \item \textbf{Auto-organisation :} Capacité des drones à se coordonner de manière autonome
    \item \textbf{Résilience validée :} Tests de tolérance aux pannes et ré-élection automatique
    \item \textbf{Performance optimisée :} Calculs distribués uniquement par le leader actuel
\end{itemize}

\textbf{Défis techniques surmontés :}
La synchronisation dans un environnement complètement distribué nécessitait une approche innovante. Notre solution de leadership par ID minimum avec timeout automatique (15 secondes) permet une coordination efficace tout en maintenant la résilience.

\textbf{Validation expérimentale :}
Les tests ont démontré que le système se réorganise automatiquement en moins de 3 secondes lors de la défaillance d'un drone, y compris du leader, validant l'architecture décentralisée.

\textbf{Recommandations pour la production :}
Pour un déploiement opérationnel, nous recommandons :
\begin{itemize}
    \item L'intégration de mécanismes de QoS MQTT pour prioriser les messages critiques
    \item L'ajout de chiffrement des communications inter-drones pour la sécurité
    \item L'implémentation de filtres de Kalman pour la prédiction de trajectoires
    \item L'utilisation de réseaux maillés (mesh) pour augmenter la portée de communication
\end{itemize}

Cette architecture décentralisée ouvre la voie à des applications critiques où la résilience est primordiale : surveillance de zones sensibles, missions de secours, ou opérations dans des environnements hostiles.