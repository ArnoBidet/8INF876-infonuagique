\section{Conclusion}

Au terme de ce projet, nous avons pu explorer les défis inhérents aux systèmes distribués appliqués à la robotique mobile.

\textbf{Ce que nous avons appris :}
La gestion de l'état global dans un système sans leader est complexe. L'approche géométrique (enveloppe convexe) permet de contourner le besoin d'un consensus lourd sur la définition de la zone.

\textbf{Problèmes rencontrés :}
La synchronisation temporelle est un enjeu majeur. Dans notre simulation, tout est local, mais dans un réseau réel (LoRa), les délais de transmission pourraient créer des "fantômes" (drones perçus à une ancienne position), faussant le calcul de la zone.

\textbf{Recommandations :}
Pour une mise en production, nous recommandons :
\begin{itemize}
    \item L'utilisation de filtres de Kalman pour prédire la position future des voisins et compenser la latence réseau.
    \item L'intégration d'un mécanisme de "Time-To-Live" (TTL) strict pour les données des voisins afin d'exclure rapidement un drone qui ne communique plus.
\end{itemize}