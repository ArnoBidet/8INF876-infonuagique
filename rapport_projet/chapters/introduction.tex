\section{Introduction}

\subsection{Introduction}
L'essor des objets connectés et des systèmes autonomes a transformé la manière dont nous interagissons avec l'espace physique. Les \textit{Location Based Services} (LBS) fournissent aujourd'hui des fonctionnalités critiques basées sur la géolocalisation des utilisateurs ou des appareils \cite{ieee2015}.

Cependant, la majorité des systèmes actuels (comme le suivi de flotte ou les applications grand public type Google Maps) reposent sur une architecture centralisée client-serveur. Dans le contexte d'une flotte de drones opérant dans des zones potentiellement déconnectées ou hostiles, cette dépendance à un serveur central pose des problèmes de latence, de résilience et de couverture réseau.

Ce projet, nommé \textbf{DroneGeoTracking}, propose une approche distribuée pour coordonner un essaim de drones, en mettant l'accent sur le maintien d'une cohésion de groupe via un géofencing dynamique et émergent.

\subsection{Description de la problématique}
La problématique principale que nous adressons est la suivante : \textit{Comment réaliser du géofencing dynamique dans un environnement entièrement décentralisé ?}

Les défis techniques identifiés sont multiples :
\begin{itemize}
    \item \textbf{Leadership distribué :} Élection automatique et ré-élection transparente des leaders sans coordination externe.
    \item \textbf{Zone émergente :} La zone de surveillance émerge dynamiquement des positions des drones actifs.
    \item \textbf{Communication peer-to-peer :} Les drones se coordonnent via MQTT sans service centralisé.
    \item \textbf{Tolérance aux pannes :} Résilience maximale face aux défaillances individuelles.
\end{itemize}

Notre objectif est de permettre aux drones de détecter s'ils s'éloignent trop du groupe (sortie de zone) ou si un voisin quitte la formation, et ce, sans aucune infrastructure au sol.

\subsection{Description du développement}
Le projet consiste en une implémentation complètement décentralisée d'un système de drones autonomes. Chaque agent exécute une logique distribuée lui permettant de :
\begin{enumerate}
    \item Publier sa position individuelle via communication MQTT.
    \item Découvrir et suivre les positions des autres drones du réseau.
    \item Participer à l'élection automatique du leader (ID minimum).
    \item Calculer la zone globale de surveillance (si leader) ou la recevoir.
    \item S'auto-réorganiser en cas de défaillance d'autres drones.
    \item Maintenir la cohésion de l'essaim sans coordination externe.
\end{enumerate}

Le système a été développé en Python pour la logique backend et la simulation, avec une interface de visualisation Web (Flask + Leaflet) pour observer le comportement de l'essaim en temps réel.
