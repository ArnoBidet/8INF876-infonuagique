\appendix
\section{Détails de l'implémentation}
Le projet est structuré comme suit :
\begin{itemize}
    \item \texttt{app.py} : Serveur Flask, simulation des drones et calcul géométrique.
    \item \texttt{templates/index.html} : Structure de la page de visualisation.
    \item \texttt{static/script.js} : Logique client, affichage de la carte Leaflet et polling de l'API.
\end{itemize}

\section{Manuel d'utilisation}
Pour lancer la démonstration via Docker :

\begin{enumerate}
    \item Assurez-vous d'avoir Docker installé.
    \item Placez-vous dans le dossier \texttt{Projet/web}.
    \item Construisez l'image :
    \begin{verbatim}
    docker build -t drone-map-demo .
    \end{verbatim}
    \item Lancez le conteneur :
    \begin{verbatim}
    docker run -p 5000:5000 drone-map-demo
    \end{verbatim}
    \item Ouvrez votre navigateur à l'adresse \texttt{http://localhost:5000}.
\end{enumerate}